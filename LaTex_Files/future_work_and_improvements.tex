\section{Future Work and Improvements}
While the current implementation of the two-wheeled self-balancing robot demonstrates robust performance, several areas can be further improved to enhance functionality, autonomy, and user experience.


\subsection{Advanced Control Strategies}
Implementing more sophisticated control algorithms, such as adaptive controllers or reinforcement learning-based approaches, could improve stability and dynamic response under varying conditions. Model predictive control (MPC) could also be explored to optimize real-time decision-making.


\subsection{Autonomous Navigation and Perception}
Enhancing the robot’s autonomy by integrating additional sensors such as LiDAR or depth cameras would enable precise environmental perception. Implementing SLAM (Simultaneous Localization and Mapping) algorithms would allow the robot to navigate and map complex environments independently.


\subsection{Seamless Mobile Integration}
Developing a dedicated mobile application with an intuitive user interface and real-time data visualization could significantly improve remote operation. Features such as wireless control, customizable movement presets, and telemetry monitoring would enhance usability.

\subsection{Optimized Power Management}
Exploring high-efficiency battery technologies, regenerative braking mechanisms, and smart power management systems could extend the robot’s operational lifespan. Additionally, integrating energy-efficient motor drivers and optimizing power consumption could further enhance performance.

By addressing these areas, the robot’s capabilities could be significantly expanded, paving the way for more advanced applications in research, automation, and real-world deployment.y management systems could extend the robot's operational time.