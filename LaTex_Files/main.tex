\documentclass{article}

% Language setting
% Replace `english' with e.g. `spanish' to change the document language
\usepackage[english]{babel}

% Set page size and margins
% Replace `letterpaper' with `a4paper' for UK/EU standard size
\usepackage[letterpaper,top=2cm,bottom=2cm,left=3cm,right=3cm,marginparwidth=1.75cm]{geometry}

% Useful packages
\usepackage{amsmath}
\usepackage{graphicx} % Required for including images
\usepackage{cite}     % Required for handling citations
\usepackage[colorlinks=true, allcolors=blue]{hyperref}

\usepackage{listings}
\usepackage{xcolor}
\usepackage{subcaption}
\usepackage{wrapfig}
\usepackage{float}

% Define colors for syntax highlighting
\definecolor{keywordcolor}{RGB}{0,0,255}
\definecolor{commentcolor}{RGB}{0,128,0}
\definecolor{stringcolor}{RGB}{255,0,0}

% Set the style for C++ code
\lstdefinestyle{cppstyle}{
	language=C++,
	keywordstyle=\color{keywordcolor},
	commentstyle=\color{commentcolor},
	stringstyle=\color{stringcolor},
	basicstyle=\ttfamily,
	numbers=left,
	numberstyle=\tiny,
	stepnumber=1,
	numbersep=5pt,
	frame=single,
	tabsize=4,
	breaklines=true
}

% Define code style
\lstdefinestyle{cppstyle2}{
  language=C++,
  basicstyle=\ttfamily\footnotesize,
  keywordstyle=\color{blue},
  commentstyle=\color{green!50!black},
  stringstyle=\color{red},
  numbers=left,
  numberstyle=\tiny, 
  stepnumber=1,
  breaklines=true,
  backgroundcolor=\color{gray!10},
  frame=single
}


\title{Self Balancing Robot}
\author{Muhammad Haris Mujeeb}

\begin{document}
\maketitle

\begin{abstract}
This report focuses on the design of a Two-wheeled Self-Balancing Robot, which embodies the classic inverted pendulum problem. Using a Kalman filter for MPU6050 sensor data fusion and implementing PID controllers. This project aims to develop a robust two-wheeled self-balancing robot that can be observed and controlled remotely. The coding for the Arduino Nano has been executed using \href{https://platformio.org/}{PlatformIO}, with the complete project available on this \href{https://github.com/haris-mujeeb/Self-Balancing-Robot}{GitHub repository}.
\end{abstract}

\thispagestyle{empty}
\tableofcontents
\listoffigures
\listoftables
\newpage

\section{Introduction}
\subsection{Background \& Motivation}
Two-wheeled vehicles are generally more agile, allowing easier navigation through tight spaces, making them ideal for congested environments. Their lighter weight and compact size facilitate easier handling while also enhancing energy efficiency. In addition, they are typically less expensive to purchase and maintain, increasing accessibility for a wider range of users.
A good base model to build such robot is \href{https://www.elegoo.com/products/elegoo-tumbller-self-balancing-robot-car}{ELEGOO Tumbler} (shown in Fig. 
\ref{fig:tumbler}), which provided nearly all the hardware required as a DIY kit.
\begin{figure}[h]
    \centering
    \includegraphics[height=6cm]{assets/tumbler.jpg}
    \caption{\label{fig:tumbler} ELEGOO Tumbler which was used for this project \cite{elegoo}.}
\end{figure}

\subsection{Project Objectives}

\subsection{Scope of Work}


\section{Literature Review}
\subsection{Overall Architecture}
Block diagram of the system
Explanation of the control flow


\subsection{Hardware Components}
\subsubsection{Microcontroller}
The ATmega328P (shown in Fig. \ref{fig:ATmega328p}) is a popular microcontroller from Microchip Technology, widely used in embedded systems and electronics projects. With a 16 MHz clock speed, 32 KB of flash memory, 2 KB of SRAM, and 1 KB of EEPROM \cite{atmega_microchip}, the ATmega328P provides ample resources for this projects application. 

\begin{figure}[h]
	\begin{subfigure}{0.5\textwidth}
	    \centering
		\includegraphics[height=3cm]{assets/ATmega328p.png}
		\caption{ATMEGA328P ANR \cite{atmega_microchip}.}
		\label{fig:ATmega328p}
	\end{subfigure}
	\begin{subfigure}{0.5\textwidth}
    	\centering
		\includegraphics[height=3cm]{assets/arduino_nano.jpg}
		\caption{Arduino Nano \cite{arduino_nano}.}
		\label{fig:arduino_nano}
	\end{subfigure}
	 \caption{ATMEGA328P MCU (a) and Arduino Nano development board (b).} % Caption for the whole figure
	\label{fig:ATmega328p_and_arduino_nano}
\end{figure}

The ATMEGA328P is also used in the Arduino Nano (shown in Fig. \ref{fig:arduino_nano}), a widely adopted development board known for its low cost and open-source ecosystem \cite{arduino_nano}. The combination of affordability and extensive community support makes it an ideal choice for rapid prototyping and academic research, ensuring easy integration with various sensors and motor drivers. 


\subsubsection{Inertial Measuring Unit}
The MPU6050 is a widely used six-axis sensor that integrates a three-axis gyroscope and a three-axis accelerometer on a single chip, making it essential for motion tracking and stabilization applications (shown in Fig. \ref{fig:mpu-6050}). Its compact design and built-in Digital Motion Processor (DMP) enable real-time processing of sensor data, which is crucial for robotics, drones, and wearable devices.
In applications like self-balancing robots, it provides accurate orientation and acceleration data necessary for maintaining stability. 
\begin{figure}[h]
    \centering
    \includegraphics[height=3cm]{assets/mpu-6050.png}
    \caption{MPU-6050 \cite{mpu6050}.}
    \label{fig:mpu-6050}
\end{figure}

\subsubsection{Motor Driver}
The TB6612FNG dual motor driver (shown in Fig. \ref{fig:tb6612fng}) allows independent control of two DC motors. It uses a MOSFET H-bridge for bidirectional and Pulse Width Modulation (PWM) based motor speed control. Fig. \ref{fig:tb6612fng_plot} shows that each channel of the TB6612FNG can deliver up to 0.85A of current continuously. Furthermore, it consists of integrated over-current protection and thermal shutdown features for enhance reliability \cite{TB6612FNG}.

\begin{figure}[h]
	\centering
	\includegraphics[height=3cm]{assets/TB6612FNG.jpg}
	\caption{TB6612FNG\cite{TB6612FNG_photo}.}
	\label{fig:tb6612fng}
\end{figure}

\begin{figure}[H]
	\centering
	\includegraphics[height=6cm]{assets/TB6612FNG_target_characteristics}
	\caption{Target characteristics for TB6612FNG \cite{TB6612FNG}.}
	\label{fig:tb6612fng_plot}
\end{figure}

Instead of using two separate MCU pins for direction control, a single pin can be used with an inverted Schmitt trigger to generate the complementary signal automatically. For this purspose SN74LVC2G14 (shown in Fig. \ref{fig:SN74LVC2G14}) is used. It also enhances motor control by improving signal stability (similar to what is shown in Fig. \ref{fig:schmitt_trigger_hysteresis}).  
\begin{figure}[h]
	\centering
	\includegraphics[height=2cm]{assets/SN74LVC2G14.png}
	\caption{SN74LVC2G14\cite{SN74LVC2G14}.}
	\label{fig:SN74LVC2G14}
\end{figure}

\begin{figure}[H]
	\centering
	\includegraphics[height=4cm]{assets/schmitt_trigger_hysteresis.png}
	\caption{Schmitt trigger output without hysteresis (left) and with hysteresis (right) \cite{schmitt_trigger_hysteresis}.}
	\label{fig:schmitt_trigger_hysteresis}
\end{figure}

\subsubsection{Drive Motors}
The drive system employs NNHYTECH GA37 520 DC motors (37mm diameter, 12V, 360 RPM) equipped with Hall effect encoders (shown in Fig. \ref{fig:dc-motor}) \cite{dc-motor}. These motors feature a reduction gearbox, which enhances torque output while maintaining controlled rotational speed, making them well-suited for applications requiring precise motion control. The Hall effect encoders generate quadrature signals, enabling accurate measurement of speed and position. Motors from NHYTech were in this case.
\begin{figure}[H]
    \centering
    \includegraphics[height=4cm]{assets/dc-motor-with-encoder.jpg}
    \caption{Similar construction motors were used in this project \cite{dc-motor}.}
    \label{fig:dc-motor}
\end{figure}

\subsubsection{Ultrasonic Distance Sensor}
The HC-SR04 is an ultrasonic distance sensor used for measuring the distance to an obstacle by sending an ultrasonic pulse and measuring the time it takes for the echo to return. It operates based on the principle of time-of-flight of sound waves, with a known speed of sound in air.
\begin{figure}[H]
    \centering
    \includegraphics[width=0.25\linewidth]{assets/Ultrasonic-HC-SR04.png}
    \caption{Ultrasonic distance sensor by  Sparkfun electronics \cite{ultra-sonic}.}
    \label{fig:ultra-sonic}
\end{figure}

\subsubsection{Infrared Sensing}
The robot is equipped with infrared proximity sensors at the front-left and front-right directions using the Everlight Elec IR Receiver (IRM-56384) and the Infrared LED (IR204C-A). These sensors detect obstacles by transmitting a modulated infrared signal and detecting its reflection.
\begin{figure}[H]
\begin{subfigure}{0.5\textwidth}
    \centering
    \includegraphics[height=3cm]{assets/HIR204C.jpg} 
    \caption{IR204C-A \cite{ir-led}.}
    \label{fig:ir-led}
\end{subfigure}
\begin{subfigure}{0.5\textwidth}
    \centering
    \includegraphics[height=3cm]{assets/ir-receiver.png} 
    \caption{IRM-56384 \cite{ir-receiver}.}
    \label{fig:ir-receiver}
\end{subfigure}
    \caption{LED-Emitter (a) and Infrared LED Reciever (b).} % Caption for the whole figure
    \label{fig:ir-sensors}
\end{figure}

\subsection{Bluetooth}
The BT16 4.2 Bluetooth transparent transmission module is a cutting-edge component that leverages the advanced capabilities of the Airoha ABI 602 single chip, which is compliant with the Bluetooth 4.2 BLE standard. This module facilitates GATT-based Bluetooth data transmission through its integrated data transparent transmission service, ensuring efficient and reliable communication. One of the key features of the BT16 module is its support for serial command mode, which enables seamless interaction between the external microcontroller unit (MCU) and the Bluetooth module. This functionality allows users to configure various parameters and exert control over the module via serial port commands. Users can modify essential settings such as the UUID, change the Bluetooth device name, and manage Bluetooth disconnection processes. 


\begin{figure}[H]
	\centering
	\includegraphics[height=3cm]{assets/BT16Module.png}
	\caption{Ultrasonic distance sensor by  Sparkfun electronics \cite{bluetooth}.}
	\label{fig:bluetooth}
\end{figure}


\subsubsection{Power Supply considerations}
This custom-designed battery box provides a portable and rechargeable power solution for the Elegoo robot \cite{battery}. Housing two 18650 LiPo batteries (likely connected in series), it delivers a regulated output to power the robot's various components. The integrated Battery Management System (BMS) ensures safe operation by providing overcharge, over-discharge, over-current, and short-circuit protection. A power switch allows for complete disconnection, while a USB charging port and status indicator LED simplify recharging and monitoring.  

\begin{figure}[H]
	\centering
	\includegraphics[height=6cm]{assets/Battery.png}
	\caption{ELEGOO battery pack with charger.}
	\label{fig:battery}
\end{figure}



\section{Mathematical Modelling}
$$
\begin{bmatrix}
    \dot{x} \\
    \ddot{x} \\
    \dot{\psi} \\
    \ddot{\psi}
\end{bmatrix}
= A
\begin{bmatrix}
    x \\
    \dot{x} \\
    \psi \\
    \dot{\psi}
\end{bmatrix}
+ B u_{input}
$$


\section{Software Implementation}
\subsection{Firmware Overview}
\begin{figure}[h]
	\centering
	\includegraphics[width=0.5\linewidth]{assets/Control_Loop.png}
	\caption{A simplified block diagram of the control loop. }
	\label{fig:control-loop}
\end{figure}

\include{sensor_fusion.tex}

\subsection{Extended Kalman Filter}
The Kalman filter provides a more sophisticated approach, estimating the true state of the system by minimizing the mean of the squared error. It involves prediction and update steps. But Kalman Filter assumes linearity in both the process and measurement models. To solve this issue we will use Extended Kalman Filter.

The Extended Kalman Filter (EKF) is an extension of the Kalman Filter for nonlinear systems, utilizing first-order Taylor series expansion to linearize process and measurement models. EKF maintains a Gaussian belief over the state, updating it through a prediction-correction cycle. The Jacobian matrices of the system dynamics and measurement functions are used to approximate state transitions and measurement updates. Its advantages include handling nonlinearities, fusing multi-sensor data, and improving estimation accuracy in noisy environments. 

\subsubsection{General State Equation}
For non-linear system, with Stochastic disturbances:
$$
\begin{aligned}
	\dot{\underline{x}}(t) &= f\left( \underline{x}(t), \underline{u}(t) \right) + \underline{d}(t) \\
	\underline{y}(t) &= h\left( \underline{x}(t) \right) + \underline{n}(t)
\end{aligned}
$$
where,
\begin{itemize}
	\item $ \dot{\underline{x}}(t) $: This represents the time derivative of the state vector $ \underline{x}(t) $, indicating how the state evolves over time.
	\item $ f $: This is a nonlinear function that describes the system dynamics, taking the current state $ \underline{x}(t) $ and the control input $ \underline{u}(t) $ as arguments. It captures how the state changes based on the current state and control inputs.
	\item $ \underline{d}(t) $: This term represents stochastic disturbances (or process noise) affecting the state dynamics, typically modeled as a zero-mean Gaussian noise.
	\item $ y(t) $: This is the measurement vector at time $ t $, representing the observed outputs of the system. It is the data collected from sensors or measurement devices.
	\item $ h $: This is a nonlinear measurement function that maps the true state vector $ \underline{x}(t) $ to the measurement space. It describes how the state influences the measurements. The function $ h $ can be complex and may involve various transformations of the state variables.
	\item $ n(t) $: This term represents measurement noise, which is also typically modeled as zero-mean Gaussian noise. It accounts for inaccuracies in the measurements due to sensor errors, environmental conditions, or other random factors that can affect the observed data.
\end{itemize}

\subsubsection{State Estimation}
For a non-linear system the state form is as follows,
$$
\begin{aligned}
	\dot{\hat{\underline{x}}}(t) &= f\left( \underline{\hat{x}}(t), \underline{u}(t) \right) + \underline{K}\left( y(t) - \hat{y}(t) \right) \\
	\hat{y}(t) &= h\left( \underline{\hat{x}}(t) \right)
\end{aligned}
$$

To compute the Kalman gain $\underline{K}$ the system must be linearized around the current state estimate. The Jacobain matrices are defined as follows:
$$
\underline{A}(t) = \frac{\mathrm{d}f}{\mathrm{d}\underline{x}} \bigg|_{\underline{\hat{x}}(t), \underline{u}(t)} \quad \text{and} \quad
\underline{C}(t) = \frac{\mathrm{d}h}{\mathrm{d}\underline{x}} \bigg|_{\underline{\hat{x}}(t)}
$$

where $\underline{A}(t)$ represents the partial derivatives of the state dynamics function $f$ and $\underline{C}(t)$ represents the partial derivatives of the measurement function $h$.

\subsubsection{Covraince Matrix}
To find a solution for the covariance matrix, we utilize Differential Riccati Equation (DRE):
$$
\dot{P}(t) = \underline{A}(t) P(t) + P(t) \underline{A}^T(t) + Q - P(t) \underline{C}^T(t) R^{-1} \underline{C}(t) P(t)
$$
where, 
\begin{itemize}
	\item $\dot{P}(t)$: his represents the time derivative of the covariance matrix $\dot{P}(t)$, which quantifies the uncertainty in the state estimate over time.
	\item $\underline{A}(t)$: This is the state transition matrix, which describes how the state evolves from one time step to the next.
	\item $Q$: This is the process noise covariance matrix, representing the uncertainty in the process model.
	\item $\underline{C}(t)$: This is the measurement matrix, which relates the state to the measurements.
	\item $R$: This is the measurement noise covariance matrix, representing the uncertainty in the measurements.
\end{itemize}

\subsubsection{Initialization}
Defining an initialization according to:
$$
P(0) = \mathbf{E}(\Delta\underline{x}(0) \Delta\underline{x}^T(0))
$$
where $P(0)$ is the initial covariance matrix, presenting the expected uncertainty in the initial state estimate. It is calculated based on the expected error in the initial state.

\subsubsection{Optimal Kalman Gain}
We can find optimal Kalman gain matrix $\underline{K}(t)$ as following:
$$
\underline{K}(t) = P(t) \underline{C}^T(t) R^{-1}
$$
It determines much the state estimate should be adjusted based on the measurement residual. It balances the uncertainty in the state estimate and the measurement noise.

\subsubsection{Time-Discrete Kalman Filter}
The time-discrete Kalman filter equations are expressed as:
$$
\begin{aligned}
	\underline{x}_{k} &= \underline{A} \underline{x}_{k-1} + \underline{B} \underline{u}_{k} + \underline{d}_{k-1} \\
	\underline{y}_{k} &= \underline{C} \underline{x}_{k} + \underline{n}_{k}
\end{aligned}
$$

where, 
\begin{itemize}
	\item $\underline{x}_k$: This is the state vector at time step $k$.
	\item $\underline{B}$: This is the control input matrix, which relates the control inputs $\underline{u}_k$ to the state.
	\item $\underline{y}_k$: This is the measurement vector at time step $k$.
	\item $\underline{d}_{k-1}$: This represents process noise at the previous time step.
	\item $\underline{n}_k$: This represents measurement noise at time step $k$.
\end{itemize}

\subsubsection{Estimated states:}
The estimated states are given by:
$$
\begin{aligned}
	\underline{\hat{x}}_{k} &= \underline{A} \  \underline{\hat{x}}_{k-1} + \underline{B} \ \underline{u}_{k} + \underline{d}_{k-1} \\
	\underline{\hat{y}}_{k} &= \underline{C} \ \underline{\hat{x}}_{k} + \underline{n}_{k}
\end{aligned}
$$
\subsubsection{State Equation for Self-Balancing Robot}
The state-space representation of the system is given by:
$$
\begin{aligned}
	\underline{\dot{x}}(t) = \underline{A}.\underline{x}(t) + \underline{B}.\underline{u}(t) \\
	\underline{y}(t) = \underline{C}.\underline{x}(t) + \underline{D}.\underline{u}(t)
\end{aligned}
$$
where the components of the State-Space Representation are,
\begin{itemize}
	\item \textbf{State Vector} $\mathbf{\underline{x}(t)}$: This vector encapsulates the internal state of the system at time t. It this case, it is defined as $\mathbf{x}_{k} = \begin{bmatrix} \text{angle} \\ \text{q\_bias} \end{bmatrix}$. Where \text{angle} represents the measured angle of the system, while \text{q\_bias} denotes the bias of the gyroscope.
	
	\item \textbf{State Transition Matrix} $\underline{\mathbf{A}}$: This matrix describes how the state evolves over time. It is defined as: $\mathbf{A}_k = \begin{bmatrix} 1 & -dt \\ 0 & 1 \end{bmatrix}$ The first row indicates that the angle is updated based on its previous value and the time step $dt$, while the second row shows that the gyroscope bias remains constant in this model.
	
	\item \textbf{Control Input Matrix} $\underline{\mathbf{B}}$: This matrix relates the control inputs to the state. In this case, it is defined as:
	$\mathbf{B}_k = 0$. This indicates that there are no direct control inputs affecting the state in this model.
	
	\item \textbf{Measurement Matrix} $\underline{\mathbf{C}}$: This matrix maps the state vector to the measurement output. It is defined as: $\mathbf{C}_k = \begin{bmatrix} 1 & 0 \end{bmatrix}$. This means that the measurement output directly reflects the angle, with no contribution from the gyroscope bias.
	
	\item \textbf{Feedforward Matrix} $\underline{\mathbf{D}}$: This matrix relates the control input directly to the measurement output. In this case, it is defined as: $\mathbf{D}_k = 0$. This indicates that there is no direct influence of the control input on the measurement output.	
\end{itemize}


\subsubsection{Measurement Noise Covariance Matrix $\mathbf{R}$}
The measurement noise variance for the angle sensor is defined as:
$$ \mathbf{R}_{k} = R_{angle} $$

\subsubsection{Process/System Noise Covariance Matrix $\mathbf{Q}$}
The process noise covariance matrix is given by:
$$
\mathbf{Q} = \begin{bmatrix} Q_{\text{angle}} & 0 \\ 0 & Q_{\text{gyro bias}} \end{bmatrix} * \Delta t
$$

\subsubsection{State Covariance Matrix $\mathbf{P}$}
The state covariance matrix is represented as:
$$
\mathbf{P}_k = \begin{bmatrix} P_{00} & P_{01} \\ P_{10} & P_{11} \end{bmatrix}
$$
\begin{itemize}
	\item $P_{00}$ represents the uncertainty in the angle estimate.
	\item $P_{11}$ represents the uncertainty in the gyroscope bias estimate.
	\item $P_{01} = P_{10}$ represent the covariance between the angle and gyroscope bias.
\end{itemize}

\subsubsection{Kalman Gain $\mathbf{K}$:}
The Kalman gain is defined as:
$$
\mathbf{K}_k = \begin{bmatrix} K_{0} \\ K_{1} \end{bmatrix}
$$

\subsubsection{Estimated states}
The estimated states are updated as follows:
$$
\theta_{measured} = \theta_{measured} + (\omega_{measured} - \omega_{bias}) * \Delta t
$$

\subsubsection{Error Calculation}
The error in the angle estimate is calculated as:
$$ \theta_{error} = \theta_{measured} - \theta_{desired} $$

\subsubsection{Time Update (prediction)}
The time update for the state covariance matrix is given by:
$$
\mathbf{P}_k = \begin{bmatrix} P_{00} & P_{01} \\ P_{10} & P_{11} \end{bmatrix}
$$
The matrix $P$ reflects the uncertainties in the angle estimate ($P_{00}$), the gyroscope bias ($P_{11}$), and the cross-covariance terms ($P_{01}$, $P_{10}$).

\subsubsection{Initialization}
The initial state covariance matrix is defined as:
$$\mathbf{P}_0 = \begin{bmatrix} 1 & 0 \\ 0 & 1 \end{bmatrix}$$
The prediction step for the covariance matrix is:
$$\mathbf{P}_k = \mathbf{A} \mathbf{P}_{k-1} \mathbf{A^T} + \mathbf{Q}$$
This expands to:
$$
\mathbf{P}_k = \begin{bmatrix} P_{00}^- + [(Q_{\text{angle}} - P_{01}^- - P_{10}^-)*\Delta t]  &  P_{01}^- - (P_{11}^-*\Delta t)  
	\\ P_{10}^- - (P_{11}^-*\Delta t)  &  P_{11}^- + (Q_{\text{gyro bias}}*\Delta t) \end{bmatrix}
$$

\subsubsection{Kalman Gain Calculation}
The Kalman gain $\underline{K}_{k}$ is computed as:
$$
\begin{aligned}
	\underline{K}_{k} &= \underline{P}_{k}^- \ \underline{C}^T ( \underline{C} \ \underline{P}_{k}^-\ \underline{C}^T  +\underline{R})^{-1} \\
	\\
	&= 
	\begin{bmatrix} P_{00} & P_{01} \\ P_{10} & P_{11} \end{bmatrix} \begin{bmatrix} 1 \\ 0 \end{bmatrix}
	\left(
	\begin{bmatrix} 1 & 0 \end{bmatrix} 
	\begin{bmatrix} P_{00} & P_{01} \\ P_{10} & P_{11} \end{bmatrix}  
	\begin{bmatrix} 1 \\ 0 \end{bmatrix} 
	+ R_{angle}
	\right)^{-1} \\ \\
	&= 
	\begin{bmatrix} P_{00} \\ P_{10}\end{bmatrix}
	\left(
	P_{00}  
	+ R_{angle}
	\right)^{-1} \\ \\
	\mathbf{K}_k &= \begin{bmatrix} \frac{ P_{00} }{ P_{00}  
			+ R_{angle}} \\ \frac{ P_{10} }{ P_{00}  
			+ R_{angle}} \end{bmatrix}
\end{aligned}
$$

\subsubsection{Measurement Update (Correction)}
The measurement update for the state covariance matrix is given by:
\[
\begin{aligned}
	\underline{P}_{k} &= \ (\underline{I} - \underline{K}_{k} \ \underline{C}) \ \underline{P}_{k}^- \\ 
	&= \ \left( \begin{bmatrix} 
		1 & 0 \\ 
		0 & 1 
	\end{bmatrix}
	- 
	\begin{bmatrix} 
		K_{0} \\  
		K_{1} 
	\end{bmatrix} 
	\begin{bmatrix} 
		1 & 0 
	\end{bmatrix}  
	\right) 
	\begin{bmatrix}
		P_{00} & P_{01} \\ 
		P_{10} & P_{11} 
	\end{bmatrix} 
	&=  
	\begin{bmatrix} 
		1 - K_{0} & 0 \\ 
		-K_{1} & 1 
	\end{bmatrix} 
	\begin{bmatrix} 
		P_{00} & P_{01} \\ 
		P_{10} & P_{11} 
	\end{bmatrix}
	&= 
	\begin{bmatrix} 
		P_{00} - K_{0} \cdot P_{00} & P_{01} - K_{0} \cdot P_{01} \\
		P_{10} - K_{1} \cdot P_{00} & P_{11} - K_{1} \cdot P_{01} \\
	\end{bmatrix}
\end{aligned}
\]

\subsection{Software Implementation of Extended Kalman Filter}
\begin{lstlisting}[style=cppstyle2]
#include <Arduino.h>

class KalmanFilter {
 private:
  float m_dt, m_Q_angle, m_Q_gyro, m_R_angle, m_C_0;
  float q_bias = 0, angle_err = 0;
  float P[2][2] = {{1, 0}, {0, 1}}; // Covariance matrix
  float K_0 = 0, K_1 = 0;

 public:
  float angle = 0;

KalmanFilter(float dt, float Q_angle, float Q_gyro, float R_angle, float C_0)
: m_dt(dt), m_Q_angle(Q_angle), m_Q_gyro(Q_gyro), m_R_angle(R_angle), m_C_0(C_0) {}

float getAngle(float measured_angle, float measured_gyro) {
	// Predict
	angle += (measured_gyro - q_bias) * m_dt;
	angle_err = measured_angle - angle;
	
	// Update covariance matrix
	P[0][0] += m_Q_angle - P[0][1] - P[1][0];
	P[0][1] -= P[1][1];
	P[1][0] -= P[1][1];
	P[1][1] += m_Q_gyro;
	
	// Compute Kalman gain
	float E = m_R_angle + m_C_0 * P[0][0];
	K_0 = (m_C_0 * P[0][0]) / E;
	K_1 = (m_C_0 * P[1][0]) / E;
	
	// Update state
	angle += K_0 * angle_err;
	q_bias += K_1 * angle_err;
	
	// Update covariance matrix
	float C0_P00 = m_C_0 * P[0][0];
	P[0][0] -= K_0 * C0_P00;
	P[0][1] -= K_0 * P[0][1];
	P[1][0] -= K_1 * P[0][0];
	P[1][1] -= K_1 * P[0][1];
	
		return angle;
	}
};
\end{lstlisting}


\section{Cascaded PID Control Loop}

The algorithm utilizes cascade Proportional–Integral–Derivative (PID) control loop to maintain balance and control the robot's movement in real time. The balancing is achieved by continuously monitoring and adjusting its state using sensor feedback. The key sensor inputs include the robot's pitch angle, gyro data and motor's encoder values, which provide information on the robot's orientation, angular velocities and position.The algorithm utilizes cascade PID control loops to maintain balance and control the robot's movement in real time. The approach includes computing control outputs for pitch, yaw, and position periodically, which are then used to adjust the motor speeds by sending corresponding pulse-width-modulation (PWM) signals to the motor driver (see Fig. \ref{fig:control-loop}).

\begin{figure}[h]
	\centering
	\includegraphics[width=0.5\linewidth]{assets/Control_Loop.png}
	\caption{A simplified block diagram of the control loop. }
	\label{fig:control-loop}
\end{figure}

As the pitch angle is more important for system stability, therefore it is updated more frequently (e.g. updated every cycle), while the PID output values for the yaw angle and position control are updated longer duration of time (e.g. after every 8th cycle).

\subsection{Basic PID Structure}
The PID controller output can be expressed mathematically as:

\begin{equation}
	u(t) = K_p e(t) + K_i \int_0^t e(\tau)d\tau + K_d \frac{d}{dt}e(t)
\end{equation}

Where:
\begin{itemize}
	\item $u(t)$ is the control signal
	\item $e(t)$ is the error signal
	\item $K_p$ is the proportional gain
	\item $K_i$ is the integral gain
	\item $K_d$ is the derivative gain
\end{itemize}



\subsection{Position Control Loop}
The outer loop manages the robot's position:

\begin{equation}
	\theta_{desired} = K_{px}(x_{desired} - x_{measured}) + K_{dx}\frac{d}{dt}(x_{desired} - x_{measured})
\end{equation}


\section{Implementation Considerations}
\subsection{Discrete Time Implementation}
For digital implementation, the PID controller is discretized:

\begin{equation}
	u[n] = K_p e[n] + K_i T_s \sum_{k=0}^n e[k] + K_d \frac{e[n] - e[n-1]}{T_s}
\end{equation}

Where $T_s$ is the sampling period.


\section{Tuning Methodology}
For tuning a general combination of Ziegler-Nichols method and practical tuning guidelines were used.
\subsection{Ziegler-Nichols Method}
The Ziegler-Nichols tuning method follows these steps:
\begin{enumerate}
	\item Set $K_i$ and $K_d$ to zero
	\item Increase $K_p$ until system oscillates with period $T_u$
	\item Record the ultimate gain $K_u$
	\item Calculate parameters:
	\begin{align*}
		K_p &= 0.6K_u \\
		T_i &= 0.5T_u \\
		T_d &= 0.125T_u
	\end{align*}
\end{enumerate}

\subsection{Practical Tuning Guidelines}
The general tuning guidelines are as follows:

\begin{itemize}
	\item Begin with small $K_p$ (= 10)
	\item Add derivative term ($K_d = 0.1K_p$)
	\item Fine-tune until stable
\end{itemize}


\subsection{Pitch PID Control:}
The pitch control loop ensures the robot maintains its upright position. The primary objective of the pitch controller is to minimize the deviation of the robot's pitch angle from a set-point, which is ideally zero degrees (i.e., upright). The pitch control output is calculated using the PD algorithm, where the error is the difference between the current pitch angle and the desired pitch angle. 
\begin{equation}
	\tau_{\theta,pid} = K_{p\theta}({\theta_{desired} - \theta_{measured}}) + K_{d\theta}\frac{d}{dt}(\theta_{desired} - \theta_{measured})
\end{equation}

Below is its code implementation:
\begin{lstlisting}[style=cppstyle]
inline void runPitchControl() {
	// Compute balance control output
	pitch_pid_output = kp_balance * (kalman.angle - 0) 
	+ kd_balance * (gyro_x  - 0);
}
\end{lstlisting}

\begin{itemize}
	\item \textbf{Proportional (P):} The proportional term is based on the current error (the pitch angle deviation from the desired value). A higher proportional gain causes the robot to respond more aggressively to larger deviations.
	\item \textbf{Derivative (D):} The derivative term anticipates future errors by considering the rate of change of the error (pitch angular velocity deviation from desired value). It provides a damping effect, which helps to reduce overshooting and oscillations.
	\item \textbf{Integral (I):} Because the system is inherently unstable when at desired upright position (steady state error can never be zero) thus adding the integral term is unnecessary. 
	
\end{itemize}

The output of the pitch PD controller ($\text{pitch\_pid\_output}$) is then used to adjust the robot's motor speeds to counteract any tilting or imbalance.

\subsection{Yaw Control:}
Yaw control is responsible for controlling the robot's rotational movement around its vertical axis. The yaw PID controller computes the control output based on the robot's angular velocity, which is measured by the gyroscope along the z-axis. The yaw control adjusts the motor speeds to achieve the desired rotational velocity, ensuring the robot maintains a stable heading.
\begin{equation}
	\tau_{\phi,pid} = K_{d\phi}\frac{d}{dt}(\phi_{desired} - \phi_{measured})
\end{equation}


Below is its code implementation:
\begin{lstlisting}[style=cppstyle]
inline void runYawControl(){
	yaw_pid_output = kd_turn * gyro_z;
}
\end{lstlisting}

Similar to pitch control, the yaw controller follows the PID principles:
\begin{itemize}
	\item \textbf{Proportional (P):} In order to calculate yaw angle, a high computational overhead is required, while yaw angle is not critical for balancing thus the proportional term is ignored.
	\item \textbf{Integral (I):} As the yaw angle is not calculated, therefore the integral term cannot be calculated as well.
	\item \textbf{Derivative (D):} The derivative term mitigates any excessive rate of change in yaw, preventing oscillations in the robot's rotation.
\end{itemize}

The yaw PID output ($\text{yaw\_pid\_output}$) is then combined with the pitch and position control outputs to compute the final motor PWM values.

\subsection{Position Control:}
Position control is implemented to ensure the robot moves smoothly and accurately along a path or to a target location. The encoder feedback from the left and right wheels is used to calculate the robot's displacement and speed. The position PID controller adjusts the motor speeds to minimize the error in position and velocity.
\begin{equation}
	\tau_{x,pid} = K_{px}(x_{desired} - x_{measured}) + K_{dx}\frac{d}{dt}(x_{desired} - x_{measured})
\end{equation}

Below is its code implementation:
\begin{lstlisting}[style=cppstyle]
inline void runPositionControl(){
	double encoder_speed = (left_encoder_position + right_encoder_position) * 0.5
	robot_position += encoder_speed;
	robot_position = constrain(robot_position, -3000, 3000);
	position_pid_output = - ki_position * robot_position - kd_position * encoder_speed;
}
\end{lstlisting}

To prevent integral windup:
\begin{equation}
	u_{limited} = \begin{cases}
		3000 & \text{if } u > 3000 \\
		u & \text{if } -3000 \leq u \leq 3000 \\
		-3000 & \text{if } u < -3000
	\end{cases}
\end{equation}

\begin{itemize}
	\item \textbf{Proportional (P):} The proportional term helps correct any immediate error in position by adjusting the motor speeds in response to the current position error.
	\item \textbf{Integral (I):} The integral term addresses any accumulated error in position that may arise due to external factors like friction or uneven terrain.
	\item \textbf{Derivative (D):} Because the position control does not require fast settling time, thus derivative term can be ignored in favour of faster calculation time.
	
\end{itemize}

The position control output ($\text{position\_pid\_output}$) is also factored into the final PWM values for motor control.

\subsection{Combining Control Outputs}

The final motor control is achieved by combining the outputs from all three PID controllers. The outputs from the pitch, yaw, and position PID controllers are used to calculate the motor speeds, which determine the robot's motion. Specifically, the following equation is used to compute the PWM values for the left and right motors:

\begin{align}
	\tau_{left,motor} &= \tau_{\theta,pid} - \tau_{\phi,pid} - \tau_{x,pid} \\
	\tau_{right,motor} &= \tau_{\theta,pid} + \tau_{\phi,pid} - \tau_{x,pid}
\end{align}

Below is its code implementation:
\begin{lstlisting}[style=cppstyle]
pwm_left = pitch_pid_output - yaw_pid_output - position_pid_output;
pwm_right = pitch_pid_output + yaw_pid_output - position_pid_output;
\end{lstlisting}

These calculated PWM values are then sent to the motor drivers to adjust the robot's movement and balance.

\subsection{Conclusion}

The use of PID controllers for pitch, yaw, and position control enables the robot to maintain balance and navigate effectively. The proportional, integral, and derivative terms in each PID loop allow the system to respond to real-time errors, minimize steady-state deviations, and anticipate future errors, leading to smooth and precise control of the robot's motion. The integration of these PID controllers is fundamental to the robot's stability and performance.



\subsection{Front Obstacle Detection}
The Ultrasonic distance measuring sensor is used to detect the obstacle in front of the robot.

\subsection{Ultrasonic Working Principle}
The sensor consists of a \textbf{transmitter} and a \textbf{receiver}:
\begin{itemize}
	\item The transmitter emits an ultrasonic pulse (40 kHz).
	\item The pulse reflects off an obstacle and is received by the receiver.
	\item The time difference between transmission and reception is used to compute the distance using the formula:
\end{itemize}
\begin{equation}
	d = \frac{t \times v}{2}
\end{equation}
where:
\begin{itemize}
	\item \(d\) is the measured distance,
	\item \(t\) is the time delay (in microseconds),
	\item \(v\) is the speed of sound (approximately 343 m/s at room temperature).
\end{itemize}

\subsubsection{Ultrasonic Implementation}
The HC-SR04 requires control signals to be sent from a microcontroller:
\begin{enumerate}
	\item A short \textbf{trigger pulse} is sent to the \texttt{TRIG} pin.
	\item The sensor responds with a high signal on the \texttt{ECHO} pin.
	\item The duration of the \texttt{ECHO} signal is measured to determine the distance.
\end{enumerate}

\subsubsection{Code for Distance Measurement}Based on the datasheet \cite{ultra-sonic}, an operating frequency of 20 Hz (corresponding to a 50 ms interval) is selected for distance measurements. The speed of sound is taken as 340.29 m/s, leading to the following constants in the code:

\begin{lstlisting}[style=cppstyle2]
	constexpr uint8_t USONIC_GET_DISTANCE_DELAY_MS = 50;
	constexpr float SPEED_OF_SOUND_HALVED = (340.29 * 100.0) / (2 * 1000 * 1000);
\end{lstlisting}

Here, the speed of sound is converted to cm/µs, and it is divided by 2 to account for the round-trip travel time of the ultrasonic pulse. The following C++ function is used to initiate distance measurement using the HC-SR04 ultrasonic sensor:
\begin{lstlisting}[style=cppstyle2]
void StartUltrasonicMeasurement() {
	if (millis() - usonicGetDistancePrevTime > USONIC_GET_DISTANCE_DELAY_MS) {
		usonicMeasureFlag = SEND;
		usonicGetDistancePrevTime = millis();
		
		attachPinChangeInterrupt(ECHO_PIN, HandleUltrasonicMeasurementInterrupt, RISING);
		
		digitalWrite(TRIG_PIN, LOW);
		digitalWrite(TRIG_PIN, HIGH);
		digitalWrite(TRIG_PIN, LOW);
	}
\end{lstlisting}

\begin{itemize}
	\item The function \texttt{StartUltrasonicMeasurement()} ensures that the measurement is taken at regular intervals.
	\item A global flag \texttt{usonicMeasureFlag} is set to \texttt{SEND}, indicating that the trigger pulse is sent.
	\item The function \texttt{attachPinChangeInterrupt()} attaches an interrupt to detect when the \texttt{ECHO} pin goes HIGH.
	\item The \texttt{TRIG} pin is first set LOW (to reset), then HIGH (to trigger the pulse), and then set LOW again.
\end{itemize}
This setup enables precise distance measurement by capturing the time delay between sending and receiving the ultrasonic pulse.


\subsubsection{Interrupt Service Routine}
The following function handles the interrupt to measure the distance:

\begin{lstlisting}[style=cppstyle2]
	void HandleUltrasonicMeasurementInterrupt() {
		if (usonicMeasureFlag == SEND) {
			usonicMeasurePrevTime = micros();
			attachPinChangeInterrupt(ECHO_PIN, HandleUltrasonicMeasurementInterrupt, FALLING);
			usonicMeasureFlag = RECEIVED;
		} else if (usonicMeasureFlag == RECEIVED) {
			usonicDistanceValue = (uint8_t)((micros() - usonicMeasurePrevTime) * SPEED_OF_SOUND_HALVED);
			usonicMeasureFlag = IDLE;
		}
	}
\end{lstlisting}

\begin{itemize}
	\item When the echo signal first rises (RISING edge), the timestamp is recorded using \texttt{micros()}.
	\item The interrupt is then reattached to detect the falling edge.
	\item When the falling edge is detected, the elapsed time is computed.
	\item The time is converted to distance using the speed of sound formula.
	\item Finally, the system resets for the next measurement.
\end{itemize}


\subsubsection{Infrared Sensing Implementation}
The IR LED transmits a modulated 38kHz infrared signal. If an obstacle is present, the signal reflects and is received by the IRM-56384, which demodulates the signal and provides a digital output. The following code is used to control and process the IR proximity sensors:
\begin{lstlisting}[style=cppstyle2]
	void IRSesorSend38KPule(unsigned char ir_pin){
		for( int i = 0; i < 39; i++) {
			digitalWrite(ir_pin, LOW);
			delayMicroseconds(9);
			digitalWrite(ir_pin, HIGH);
			delayMicroseconds(9);
		}
	}
	
	void ProcessLeftIRSensor(){
		if (millis() - irLeftCountTime > IR_COUNT_DELAY_MS) {
			UpdateSlidingWindow(irLeftPulseCount >= 3, irLeftHistory, irLeftIndex, irLeftRunningCount);    
			irLeftIsObstacle = (irLeftRunningCount >= 5);
			irLeftPulseCount = 0;
			irLeftCountTime = millis();
		}
	}
	
	void ProcessRightIRSensor(){
		if (millis() - irRightCountTime > IR_COUNT_DELAY_MS) {
			UpdateSlidingWindow((irRightPulseCount >= 3), irRightHistory, irRightIndex, irRightRunningCount);
			irRightIsObstacle = (irRightRunningCount >= 5);
			irRightPulseCount = 0;
			irRightCountTime = millis();
		}
	}
\end{lstlisting}

\section{Remote Control and Communication}

The remote control and communication system of the two-wheeled self-balancing robot was designed to enable seamless and reliable interaction between the user and the robot. For this purpose, we integrated the \textbf{BT16 Bluetooth UART Module}, which provided a robust wireless communication link.

\subsection{Bluetooth Module Integration}
The \textbf{BT16 Bluetooth UART Module} was chosen due to its compatibility with microcontrollers and its ability to provide stable serial communication over Bluetooth. The module was interfaced with the microcontroller using UART communication, ensuring efficient data transmission and reception.

\subsection{Communication Protocol}
A custom communication protocol was developed to manage the exchange of control commands and telemetry data. Key features of the protocol included:

\begin{itemize}
\item \textbf{Command Transmission:} The robot could receive commands for movement control, mode switching, and parameter adjustments from a remote device.
\item \textbf{Telemetry Feedback:} The robot transmitted real-time data such as tilt angle, battery status, and motor performance back to the controlling device.
\item \textbf{Error Handling:} Mechanisms were implemented to detect and recover from communication errors, ensuring reliable operation even in noisy environments.
\end{itemize}

\subsection{User Interface for Remote Control}
The Bluetooth communication enabled the development of a user-friendly interface for remote control. This interface could be a mobile application or a desktop-based GUI, allowing users to interact with the robot intuitively. Features included:

\begin{itemize}
\item \textbf{Joystick Control:} For real-time maneuvering of the robot.
\item \textbf{Status Monitoring:} Display of critical telemetry data.
\item \textbf{Parameter Tuning:} On-the-fly adjustment of control parameters for performance optimization.
\end{itemize}

\subsection{Testing and Performance}
Extensive testing was conducted to ensure the reliability and responsiveness of the Bluetooth communication system. The tests focused on:

\begin{itemize}
\item \textbf{Range:} Assessing the effective communication distance.
\item \textbf{Latency:} Measuring the delay between command transmission and robot response.
\item \textbf{Stability:} Ensuring consistent performance in various environments.
\end{itemize}

The integration of the Bluetooth module significantly enhanced the robot's usability, providing a flexible and responsive remote control solution that contributed to the overall functionality and user experience of the system.

\include{simulation_and_testing.tex}
\include{power_management.tex}
\section{User Interface}

The user interface was designed to provide intuitive control and comprehensive feedback to the operator. Both hardware and software interfaces were developed to enhance the user experience.

\subsection{Graphical User Interface (GUI)}
A GUI was created for both desktop and mobile platforms, leveraging Bluetooth connectivity for communication. The interface included:

\begin{itemize}
\item \textbf{Real-Time Monitoring:} Display of sensor data such as tilt angle, speed, and battery status.
\item \textbf{Control Inputs:} Virtual joystick and buttons for maneuvering the robot.
\item \textbf{Parameter Adjustment:} Sliders and input fields for tuning control parameters.
\end{itemize}

\subsection{Physical Interface}
For manual control, physical buttons and switches were integrated on the robot's chassis. These controls allowed for quick access to basic functions like power on/off and mode switching.
\section{Safety Features}

Ensuring the safety of both the robot and its environment was a priority in the design process. Multiple safety features were integrated into the system.

\subsection{Emergency Stop Mechanism}
An emergency stop function was implemented, allowing the robot to be immediately powered down via a physical button or remote command in case of malfunction or hazardous situations.

\subsection{Overcurrent and Overvoltage Protection}
Electronic protection circuits were included to safeguard the motors and microcontroller from electrical faults. These circuits automatically disconnected power in the event of overcurrent or overvoltage conditions.

\subsection{Fall Detection and Recovery}
Sensors were programmed to detect when the robot tipped beyond a recoverable angle. In such cases, the motors were disabled to prevent damage, and a recovery sequence could be initiated to return the robot to an upright position.
\section{Future Work and Improvements}

While the current implementation of the two-wheeled self-balancing robot achieved significant milestones, several areas for future improvement were identified.

\subsection{Enhanced Control Algorithms}
Exploring advanced control techniques such as adaptive control and machine learning-based approaches could further enhance the robot's stability and performance.

\subsection{Autonomous Navigation}
Integrating additional sensors like LiDAR and implementing SLAM (Simultaneous Localization and Mapping) algorithms would enable the robot to navigate autonomously in complex environments.

\subsection{Mobile App Development}
Developing a dedicated mobile application with enhanced features and a more user-friendly interface would improve the overall user experience.

\subsection{Extended Battery Life}
Investigating alternative power sources or more efficient battery management systems could extend the robot's operational time.
\section{Real-World Applications}

The technology and design principles behind the two-wheeled self-balancing robot have numerous real-world applications.

\subsection{Personal Transportation}
Self-balancing technology is commonly used in personal transport devices like hoverboards and Segways, offering convenient and efficient mobility solutions.

\subsection{Robotics Research and Education}
This project serves as a valuable educational tool for understanding control systems, sensor fusion, and robotics, making it suitable for academic and research purposes.

\subsection{Delivery and Logistics}
Self-balancing robots can be adapted for use in delivery services, particularly in navigating urban environments where maneuverability is crucial.

\subsection{Assistive Technology}
In healthcare, self-balancing robots can assist individuals with mobility impairments, providing stability and support in daily activities.
\include{challenges_and_limitations.tex}

\bibliographystyle{IEEEtran}
\bibliography{references.bib}

\end{document}