
\subsection{Front Obstacle Detection}
The Ultrasonic distance measuring sensor is used to detect the obstacle in front of the robot.

\subsection{Ultrasonic Working Principle}
The sensor consists of a \textbf{transmitter} and a \textbf{receiver}:
\begin{itemize}
	\item The transmitter emits an ultrasonic pulse (40 kHz).
	\item The pulse reflects off an obstacle and is received by the receiver.
	\item The time difference between transmission and reception is used to compute the distance using the formula:
\end{itemize}
\begin{equation}
	d = \frac{t \times v}{2}
\end{equation}
where:
\begin{itemize}
	\item \(d\) is the measured distance,
	\item \(t\) is the time delay (in microseconds),
	\item \(v\) is the speed of sound (approximately 343 m/s at room temperature).
\end{itemize}

\subsubsection{Ultrasonic Implementation}
The HC-SR04 requires control signals to be sent from a microcontroller:
\begin{enumerate}
	\item A short \textbf{trigger pulse} is sent to the \texttt{TRIG} pin.
	\item The sensor responds with a high signal on the \texttt{ECHO} pin.
	\item The duration of the \texttt{ECHO} signal is measured to determine the distance.
\end{enumerate}

\subsubsection{Code for Distance Measurement}Based on the datasheet \cite{ultra-sonic}, an operating frequency of 20 Hz (corresponding to a 50 ms interval) is selected for distance measurements. The speed of sound is taken as 340.29 m/s, leading to the following constants in the code:

\begin{lstlisting}[style=cppstyle2]
	constexpr uint8_t USONIC_GET_DISTANCE_DELAY_MS = 50;
	constexpr float SPEED_OF_SOUND_HALVED = (340.29 * 100.0) / (2 * 1000 * 1000);
\end{lstlisting}

Here, the speed of sound is converted to cm/µs, and it is divided by 2 to account for the round-trip travel time of the ultrasonic pulse. The following C++ function is used to initiate distance measurement using the HC-SR04 ultrasonic sensor:
\begin{lstlisting}[style=cppstyle2]
void StartUltrasonicMeasurement() {
	if (millis() - usonicGetDistancePrevTime > USONIC_GET_DISTANCE_DELAY_MS) {
		usonicMeasureFlag = SEND;
		usonicGetDistancePrevTime = millis();
		
		attachPinChangeInterrupt(ECHO_PIN, HandleUltrasonicMeasurementInterrupt, RISING);
		
		digitalWrite(TRIG_PIN, LOW);
		digitalWrite(TRIG_PIN, HIGH);
		digitalWrite(TRIG_PIN, LOW);
	}
\end{lstlisting}

\begin{itemize}
	\item The function \texttt{StartUltrasonicMeasurement()} ensures that the measurement is taken at regular intervals.
	\item A global flag \texttt{usonicMeasureFlag} is set to \texttt{SEND}, indicating that the trigger pulse is sent.
	\item The function \texttt{attachPinChangeInterrupt()} attaches an interrupt to detect when the \texttt{ECHO} pin goes HIGH.
	\item The \texttt{TRIG} pin is first set LOW (to reset), then HIGH (to trigger the pulse), and then set LOW again.
\end{itemize}
This setup enables precise distance measurement by capturing the time delay between sending and receiving the ultrasonic pulse.


\subsubsection{Interrupt Service Routine}
The following function handles the interrupt to measure the distance:

\begin{lstlisting}[style=cppstyle2]
	void HandleUltrasonicMeasurementInterrupt() {
		if (usonicMeasureFlag == SEND) {
			usonicMeasurePrevTime = micros();
			attachPinChangeInterrupt(ECHO_PIN, HandleUltrasonicMeasurementInterrupt, FALLING);
			usonicMeasureFlag = RECEIVED;
		} else if (usonicMeasureFlag == RECEIVED) {
			usonicDistanceValue = (uint8_t)((micros() - usonicMeasurePrevTime) * SPEED_OF_SOUND_HALVED);
			usonicMeasureFlag = IDLE;
		}
	}
\end{lstlisting}

\begin{itemize}
	\item When the echo signal first rises (RISING edge), the timestamp is recorded using \texttt{micros()}.
	\item The interrupt is then reattached to detect the falling edge.
	\item When the falling edge is detected, the elapsed time is computed.
	\item The time is converted to distance using the speed of sound formula.
	\item Finally, the system resets for the next measurement.
\end{itemize}


\subsubsection{Infrared Sensing Implementation}
The IR LED transmits a modulated 38kHz infrared signal. If an obstacle is present, the signal reflects and is received by the IRM-56384, which demodulates the signal and provides a digital output. The following code is used to control and process the IR proximity sensors:
\begin{lstlisting}[style=cppstyle2]
	void IRSesorSend38KPule(unsigned char ir_pin){
		for( int i = 0; i < 39; i++) {
			digitalWrite(ir_pin, LOW);
			delayMicroseconds(9);
			digitalWrite(ir_pin, HIGH);
			delayMicroseconds(9);
		}
	}
	
	void ProcessLeftIRSensor(){
		if (millis() - irLeftCountTime > IR_COUNT_DELAY_MS) {
			UpdateSlidingWindow(irLeftPulseCount >= 3, irLeftHistory, irLeftIndex, irLeftRunningCount);    
			irLeftIsObstacle = (irLeftRunningCount >= 5);
			irLeftPulseCount = 0;
			irLeftCountTime = millis();
		}
	}
	
	void ProcessRightIRSensor(){
		if (millis() - irRightCountTime > IR_COUNT_DELAY_MS) {
			UpdateSlidingWindow((irRightPulseCount >= 3), irRightHistory, irRightIndex, irRightRunningCount);
			irRightIsObstacle = (irRightRunningCount >= 5);
			irRightPulseCount = 0;
			irRightCountTime = millis();
		}
	}
\end{lstlisting}
